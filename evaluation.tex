\section{Evaluation}
\label{sec.evaluation}

To demonstrate that our ``popular paths'' metric is useful and practical, 
we use our Lind prototype as a testing tool. 
To evaluate the effectiveness of the ``popular paths'' metric,
%in containing untrusted code and protecting the OS kernel,
we compared Lind against three existing
virtualization systems -- Docker, LXC, and Graphene.
%This section describes
%the purpose and setup of our experiments, and presents and discusses our results.
We chose these three systems because they currently represent the most
widely-used VM design models for securing the OS kernel.
LXC is a well-known container designed specifically for the Linux kernel.
Docker is a widely-used container that wraps an application in a self-contained filesystem, while
Graphene is an open source library OS designed to run an application in a virtual machine environment.
Lastly, we also tested Native Linux to serve as a
baseline for comparison.
%
Our tests were designed to answer four fundamental questions:

\textit{How does Lind compare to other virtualization systems
in protecting against zero-day Linux kernel bugs?}
(Section~{\ref{Linux-Kernel-Bug-Test-and-Evaluation}})

\textit{How much of the underlying kernel code is exposed, and is thus
vulnerable in different virtualization systems?}
(Section~{{\ref{Reachable-Kernel-Trace-Analysis-for-Different-Virtualization-Systems}})

\textit{If Lind's SafePOSIX construction has bugs, how severe an impact would
this vulnerability have?}
(Section~{{\ref{Reachable-Kernel-Trace-Analysis-for-Repy-Sandbox}})

\textit{In the Lind prototype, what would be the expected performance overhead in
real-world applications? Can developers make use of the ``popular paths'' metric to develop 
practical systems?}
(Section~{{\ref{Performance-Evaluation}})

\subsection{Linux Kernel Bug Test and Evaluation}
\label{Linux-Kernel-Bug-Test-and-Evaluation}

%\textbf{Test Purpose.}
%To evaluate how well each virtualization system protects the Linux kernel
%against reported zero-day bugs.
%\brendan{This is my best understanding of how the tests were conducted. Yiwen et al., please jump in and correct me if I got something wrong.}
%\yiwen{The setup section looks good to me.}

\noindent
\textbf{Setup.}
To evaluate how well each virtualization system protects the Linux kernel
against reported zero-day bugs,
we examined a list of 69 historical bugs that had been identified and patched in version 3.14.1 of the Linux kernel \cite{CVE-Datasource}. 
We obtained this list of 69 bugs from National Vulnerability Database (NVD) \cite{NVD}. 
This list was an exhaustive list of CVE \cite{CVE} bugs that had been confirmed as present in this Linux kernel version at the time of our study (September, 2015). 
By analyzing security patches for those bugs,
we were able to identify the lines of code in the kernel that correspond to each one.

In the following evaluation, we assume that a bug is potentially triggerable if the lines of code that were changed in the patch are reached
(i.e., the same metric described in Section~\ref{sec.metric}).
This measure may overestimate potential danger posed by a system since simply reaching the buggy code does not mean that guest code
actually has enough control to exploit the bug.
However, this overestimate should apply equally to all of the systems we tested, which means it is still a useful method of comparison.

Next, we sought out proof-of-concept code that could trigger each bug.
We were able to obtain or create code to trigger nine out of the 69 bugs \cite{Exploit-Database}.
For the rest, we used the Trinity system call fuzzer
\cite{Trinity} on Linux 3.14.1 (referred to as ``Native'' Linux in Table~\ref{table:trigger_vulnerabilities}).
By comparing the code reached during fuzzing with the lines of code affected by security patches,
we were able to identify an additional 26 bugs that could be triggered. 
All together, we identified a total of 35 bugs that we were able to trigger from the user space, and these formed our final dataset for the evaluation. 

We then evaluated the protection afforded by four virtualization systems (including Lind) by attempting to trigger the 35 bugs from inside each one.
The host system for each test ran a version of Linux 3.14.1 with gcov instrumentation enabled.
For the nine bugs that we could trigger directly, we ran the proof of concept exploit inside the guest.
For the other 26, we ran the Trinity fuzzer inside the guest, exercising each system call 1,000,000 times with random inputs.
Finally, we checked whether the lines of code containing each bug were reached in the host kernel,
indicating that the guest could have triggered the bug.

\noindent
\textbf{Results.}
We found that a substantial number of bugs could be triggered in existing
virtualization systems, as shown in Table \ref{table:trigger_vulnerabilities}.
All (100\%) bugs were triggered in Native Linux,
while the other programs had lower rates: 8/35 (22.9\%)  in Docker,
12/35 (34.3\%)  in LXC, and 8/35 (22.9\%) bugs in Graphene.
In comparison, only 1 out of 35 bugs  (2.9\%) was triggered in Lind.

When we take a closer look at the results, we can see that these outcomes
have a lot to do with the design principles of these virtualization systems and
the way in which they handle system call requests.
Graphene \cite{Graphene-14} is a library OS that relies heavily on the Linux kernel to handle system calls.
Graphene's Linux library implements the Linux system calls using a variant of the
Drawbridge \cite{Drawbridge-11} ABI, which has 43 functions. Those ABI functions
are provided by the Platform Adaptation Layer (PAL), implemented using 50 calls
to the kernel. It turns out that 8 vulnerabilities in our test were triggered by PAL's
50 system calls. On the contrary, Lind only relies on 33 system calls, which significantly reduces
risks and avoids 7 out of the 8 bugs.

Graphene supports many complex and risky system calls, such as \texttt{execve}, \texttt{msgsnd}, and \texttt{futex},
that reached the risky (unpopular) portion of the kernel and eventually led to kernel bugs.
In addition, for many basic and frequently-used system calls like \texttt{open} and \texttt{read},
Graphene allows rarely-used flags and arguments to be passed down to the kernel, which triggered bugs in
the unpopular paths.
In Lind, all system calls only allow a restricted set of simple and frequently-used flags and arguments.
One example from our test result is that Graphene allows \texttt{O\_TMPFILE} flag to pass down with
\texttt{path\_openat()} system call. This reached risky lines of code inside \texttt{fs/namei.c} in the kernel,
and eventually triggered bug CVE-2015-5706.
The same bug was triggered in the same way inside Docker and LXC, but was successfully prevented by Lind,
due to its strict control on flags and arguments.
In fact, the design of Graphene requires extensive interaction
with the host kernel and, hence, has many risks. The developers of Graphene manually conducted
an analysis of 291 Linux vulnerabilities from 2011 to 2013, and found out that Graphene's design can not prevent 144 of those vulnerabilities.

LXC \cite{LXC} is an operating-system-level virtualization container that uses Linux kernel features to achieve containment.
Docker \cite{Docker} is a Linux container that runs on top of LXC. The two containers have very similar design features
that both rely directly on the Linux kernel to handle system call requests. Since system calls inside Docker are passed down
to LXC and then into the kernel, we found out that all 8 kernel vulnerabilities triggered inside Docker were also triggered
with LXC. In addition, LXC interacts with the kernel via its \texttt{liblxc} library component, which triggered the extra 4 bugs.

It should be noted that although the design of Lind only accesses popular paths in the kernel and implements SafePOSIX inside
of a sandbox, there are a few fundamental building blocks for which Lind must rely on the kernel. To be more specific,
\texttt{mmap} and \texttt{threads} cannot be recreated inside SafePOSIX without interaction with the kernel, since there has
to be some basic operations to access the hardware. Therefore, in our design of Lind, \texttt{mmap} and \texttt{threads}
are passed down to the kernel, and any vulnerabilities related to them are  unavoidable.
CVE-2014-4171 is a bug triggered by \texttt{mmap} inside Lind. It was also triggered inside Docker, LXC, and Graphene, indicating
that those systems rely on the kernel to perform \texttt{mmap} operations as well.

Our initial results suggest that bugs are usually triggered by extensive interaction with the unpopular paths in the kernel through
complex system calls, or basic system calls
with complicated or rarely used flags. The \emph{Lock-in-Pop} design, and thus Lind,
provides strictly controlled access to the kernel, and so, poses
the least risk.

%To test if a bug could be triggered, we created or located C
%code capable of exploiting each kernel bug \cite{Exploit-Database}.
%We were only able to trigger and obtain results for 35 out of the 69 bugs in our
%experiments. In some cases, there was a
%difficulty in clearly determining if triggering had occurred, while in other we
%were unable to find code to trigger them. We decided to focus our study on
%this group of 35 bugs and leave the more complex ones to future work.

%We compiled and ran the exploit C code under each virtualization system to
%obtain their kernel traces, and then used our kernel trace safety metric to
%determine if a specific bug was triggered.

%\begin{table*}[!ht]
%\begin{table}[tbp]
%\scriptsize
\begin{table}[h]
\scriptsize
\centering

\begin{tabular}{|p{1.7cm}|p{.6cm}|p{.65cm}|p{.65cm}|p{.9cm}|p{.6cm}|}\hline

\multirow{2}{1.7cm}{\bf Vulnerability}    &  \multirow{2}{.6cm}{\bf Native Linux}
 & \multirow{2}{.6cm}{\bf Docker} & \multirow{2}{.6cm}{\bf LXC} &
\multirow{2}{1cm}{\bf Graphene} & \multirow{2}{.6cm}{\bf Lind} \\
& & & & & \\
\hline

 CVE-2015-5706 & \multirow{1}{.7cm}{{\color{red}\ding{51}}} &
 \multirow{1}{1cm}{{\color{red}\ding{51}}} &
\multirow{1}{1cm}{{\color{red}\ding{51}}} &
\multirow{1}{1cm}{{\color{red}\ding{51}}} &
\ding{55}  \\

 CVE-2015-0239 & \multirow{1}{.7cm}{{\color{red}\ding{51}}} &
 \ding{55} & \multirow{1}{1cm}{{\color{red}\ding{51}}} &
 \ding{55}  & \ding{55}  \\

 CVE-2014-9584 & \multirow{1}{.7cm}{{\color{red}\ding{51}}} &
 \ding{55} & \ding{55} &
 \ding{55}  & \ding{55}  \\

 CVE-2014-9529 & \multirow{1}{.7cm}{{\color{red}\ding{51}}} &
 \ding{55} & \multirow{1}{1cm}{{\color{red}\ding{51}}} &
\ding{55}  & \ding{55}  \\

 CVE-2014-9322 & \multirow{1}{.7cm}{{\color{red}\ding{51}}} &
\multirow{1}{1cm}{{\color{red}\ding{51}}} & \multirow{1}{1cm}{{\color{red}\ding{51}}} &
\multirow{1}{1cm}{{\color{red}\ding{51}}}  & \ding{55}
\\

 CVE-2014-9090 & \multirow{1}{.7cm}{{\color{red}\ding{51}}} &
 \ding{55} & \ding{55} &
 \ding{55}  & \ding{55}  \\

 CVE-2014-8989 & \multirow{1}{.7cm}{{\color{red}\ding{51}}} &
 \multirow{1}{1cm}{{\color{red}\ding{51}}} &
\multirow{1}{1cm}{{\color{red}\ding{51}}} &
\multirow{1}{1cm}{{\color{red}\ding{51}}} &
\ding{55}  \\

 CVE-2014-8559 & \multirow{1}{.7cm}{{\color{red}\ding{51}}} &
 \ding{55} & \ding{55} &
 \ding{55}  & \ding{55}  \\

 CVE-2014-8369 & \multirow{1}{.7cm}{{\color{red}\ding{51}}} &
 \ding{55} & \ding{55} &
 \ding{55}  & \ding{55}  \\

 CVE-2014-8160 & \multirow{1}{.7cm}{{\color{red}\ding{51}}} &
 \ding{55} & \multirow{1}{1cm}{{\color{red}\ding{51}}} &
\ding{55}  & \ding{55}  \\

 CVE-2014-8134 & \multirow{1}{.7cm}{{\color{red}\ding{51}}} &
 \ding{55} & \multirow{1}{1cm}{{\color{red}\ding{51}}} &
\multirow{1}{1cm}{{\color{red}\ding{51}}}  & \ding{55}
\\

 CVE-2014-8133 & \multirow{1}{.7cm}{{\color{red}\ding{51}}} &
 \ding{55} & \ding{55} &
\ding{55}  & \ding{55}  \\

 CVE-2014-8086 & \multirow{1}{.7cm}{{\color{red}\ding{51}}} &
 \multirow{1}{1cm}{{\color{red}\ding{51}}} &
\multirow{1}{1cm}{{\color{red}\ding{51}}} &
\ding{55} & \ding{55}  \\

 CVE-2014-7975 & \multirow{1}{.7cm}{{\color{red}\ding{51}}} &
 \ding{55} & \ding{55} &
 \ding{55}  & \ding{55}  \\

 CVE-2014-7970 & \multirow{1}{.7cm}{{\color{red}\ding{51}}} &
 \ding{55} & \ding{55} &
 \ding{55}  & \ding{55}  \\

 CVE-2014-7842 & \multirow{1}{.7cm}{{\color{red}\ding{51}}} &
 \ding{55} & \ding{55} &
 \ding{55}  & \ding{55}  \\

 CVE-2014-7826 & \multirow{1}{.7cm}{{\color{red}\ding{51}}} &
 \ding{55} & \ding{55}  &
\multirow{1}{1cm}{{\color{red}\ding{51}}}  & \ding{55}
\\

 CVE-2014-7825 & \multirow{1}{.7cm}{{\color{red}\ding{51}}} &
 \ding{55} & \ding{55} &
\multirow{1}{1cm}{{\color{red}\ding{51}}}  & \ding{55}
\\

 CVE-2014-7283 & \multirow{1}{.7cm}{{\color{red}\ding{51}}} &
 \ding{55} & \ding{55} &
 \ding{55}  & \ding{55}  \\

 CVE-2014-5207 & \multirow{1}{.7cm}{{\color{red}\ding{51}}} &
 \ding{55} & \ding{55} &
 \ding{55}  & \ding{55}  \\

 CVE-2014-5206 & \multirow{1}{.7cm}{{\color{red}\ding{51}}} &
 \multirow{1}{1cm}{{\color{red}\ding{51}}} &
\multirow{1}{1cm}{{\color{red}\ding{51}}} &
\ding{55}  & \ding{55}
\\

 CVE-2014-5045 & \multirow{1}{.7cm}{{\color{red}\ding{51}}} &
 \ding{55} & \ding{55} &
 \ding{55}  & \ding{55}  \\

 CVE-2014-4943 & \multirow{1}{.7cm}{{\color{red}\ding{51}}} &
 \ding{55} & \ding{55} &
 \ding{55}  & \ding{55}  \\

 CVE-2014-4667 & \multirow{1}{.7cm}{{\color{red}\ding{51}}} &
 \ding{55} & \ding{55} &
 \multirow{1}{1cm}{{\color{red}\ding{51}}}  & \ding{55}  \\

 CVE-2014-4508 & \multirow{1}{.7cm}{{\color{red}\ding{51}}} &
 \ding{55} & \ding{55} &
 \ding{55}  & \ding{55}  \\

 CVE-2014-4171 & \multirow{1}{.7cm}{{\color{red}\ding{51}}} &
 \multirow{1}{1cm}{{\color{red}\ding{51}}} &
\multirow{1}{1cm}{{\color{red}\ding{51}}} &
\multirow{1}{1cm}{{\color{red}\ding{51}}} &
\multirow{1}{1cm}{{\color{red}\ding{51}}}  \\

 CVE-2014-4157 & \multirow{1}{.7cm}{{\color{red}\ding{51}}} &
 \ding{55} & \ding{55} &
 \ding{55}  & \ding{55}  \\

 CVE-2014-4014 & \multirow{1}{.7cm}{{\color{red}\ding{51}}} &
 \multirow{1}{1cm}{{\color{red}\ding{51}}} &
\multirow{1}{1cm}{{\color{red}\ding{51}}} &
\ding{55}  & \ding{55}
\\

 CVE-2014-3940 & \multirow{1}{.7cm}{{\color{red}\ding{51}}} &
 \multirow{1}{1cm}{{\color{red}\ding{51}}} & \multirow{1}{1cm}{{\color{red}\ding{51}}} &
\ding{55}  & \ding{55}  \\

 CVE-2014-3917 & \multirow{1}{.7cm}{{\color{red}\ding{51}}} &
 \ding{55} & \ding{55} &
\ding{55}  & \ding{55}  \\

 CVE-2014-3153 & \multirow{1}{.7cm}{{\color{red}\ding{51}}} &
 \ding{55} & \ding{55} &
  \ding{55}  & \ding{55}  \\

 CVE-2014-3144 & \multirow{1}{.7cm}{{\color{red}\ding{51}}} &
 \ding{55} & \ding{55} &
 \ding{55}  & \ding{55}  \\

 CVE-2014-3122 & \multirow{1}{.7cm}{{\color{red}\ding{51}}} &
 \ding{55} & \ding{55} &
 \ding{55}  & \ding{55}  \\

 CVE-2014-2851 & \multirow{1}{.7cm}{{\color{red}\ding{51}}} &
 \ding{55} & \ding{55} &
 \ding{55}  & \ding{55}  \\

 CVE-2014-0206 & \multirow{1}{.7cm}{{\color{red}\ding{51}}} &
 \ding{55} & \ding{55} &
 \ding{55}  & \ding{55}  \\
\hline

 {\bf Vulnerabilities Triggered} & \multirow{2}{1cm}{\bf 35/35 (100\%)} & {\bf 8/35 (22.9\%)} &
 {\bf 12/35 (34.3\%)} &
 {\bf 8/35 (22.9\%)}  & {\bf 1/35 (2.9\%)}  \\
\hline
\end{tabular}

\caption {\small Linux kernel bugs, and vulnerabilities in different virtualization systems
({\color{red}\ding{51}}: vulnerability triggered;
\ding{55}: vulnerability not triggered).}

\label{table:trigger_vulnerabilities}
\end{table}

\subsection{Comparison of Kernel Code Exposure in Different Virtualization
Systems}
\label{Reachable-Kernel-Trace-Analysis-for-Different-Virtualization-Systems}
\begin{table}
\centering
\scriptsize
\begin{tabular}{|l|l|l|l|l|}
  \hline
  \multirow{3}{1.5cm}{\bf Virtualization system} & \multirow{3}{0.5cm}{\bf \# of Bugs} & \multicolumn{3}{c|}{\bf Kernel trace (LOC)} \\ \cline{3-5}
  & & \multirow{2}{1.2cm}{Total coverage} & \multirow{2}{1.2cm}{In popular paths} & \multirow{2}{1.2cm}{In risky paths}  \\
  & & & & \\  \hline
  LXC & 12 & 127.3K & 70.9K & 56.4K \\
  \hline
  Docker & 8 & 119.0K & 69.5K & 49.5K \\
  \hline
  Graphene & 8 & 95.5K & 62.2K & 33.3K \\
  \hline
  Lind & 1 & 70.3K & 70.3K & 0 \\
  \hline
\end{tabular}
\caption{\small Reachable kernel trace analysis for different virtualization
systems.}
\label{table:trace-systems}
\end{table}

%\textbf{Test Purpose.}
%To determine how much of the underlying kernel can be executed and exposed in
%each system. This helps us understand the potential risks a virtualization system
%may pose based upon how much access it allows to the kernel code.

\noindent
\textbf{Setup.}
%To analyze the reachable kernel paths for each
%virtualization system,
To determine how much of the underlying kernel can be executed and exposed in
each system,
we conducted system call fuzzing with Trinity (similar to our approach in
Section~{\ref{sec.metric}}) to obtain
kernel traces. This helps us understand the potential risks a virtualization system
may pose based upon how much access it allows to the kernel code.
All experiments were conducted under Linux kernel 3.14.1.

\noindent
\textbf{Results.}
We obtained the total reachable kernel trace for
each tested system,
and further analyzed the components of those traces. These results,
shown in Table \ref{table:trace-systems}, affirm that Lind accessed the
least amount of code in the OS
kernel. More importantly, all the kernel code it did access was in the
popular kernel paths that contain fewer bugs (Section~{\ref{Verification-of-Hypothesis}}).
A large portion of the kernel paths accessed by Lind lie in
\texttt{fs/} to perform file system operations.
To restrict file system calls to popular paths, Lind allows only basic calls,
like \texttt{open()}, \texttt{close()}, \texttt{read()}, \texttt{write()}, \texttt{mkdir()},
\texttt{rmdir()}, and permits only commonly-used flags like \texttt{O\_CREAT}, \texttt{O\_EXCL},
 \texttt{O\_APPEND}, \texttt{O\_TRUNC},
\texttt{O\_RDONLY}, \texttt{O\_WRONLY}, and \texttt{O\_RDWR}
for \texttt{open()}.

The other virtualization systems all accessed a substantial number of code
paths in the kernel, and they all accessed a larger section from the unpopular
paths. This is because they rely on the underlying host kernel to implement
complex functionality. Therefore, they are more dependent on complex system
calls, and allow extensive use of complicated flags.  For example, Graphene's
system call API supports multiple processes via \texttt{fork()} and signals,
and therefore accesses many risky lines of code. 
For basic and frequently-used system calls like \texttt{open},
Graphene allows rarely-used flags, such as \texttt{O\_TMPFILE} and \texttt{O\_NONBLOCK}
to pass down to the kernel, thus reaching risky lines in the kernel that could lead to bugs.
By default, Docker and LXC do not wrap or filter system calls made by applications running in a container.
Thus, programs have access to basically all the system calls, and rarely used flags, such as \texttt{O\_TMPFILE},
\texttt{O\_NONBLOCK}, and \texttt{O\_DSYNC}. Again, this mean they can reach risky lines of code in the kernel.

To summarize, our analysis suggests that Lind triggers the fewest kernel bugs because
it has better control over the portions of the OS kernel accessed by applications.

\subsection{Impact of Potential Vulnerabilities in Lind's SafePOSIX Re-creation}
\label{Reachable-Kernel-Trace-Analysis-for-Repy-Sandbox}

%\textbf{Test Purpose.}
%To understand what potential security risks could be posed if Lind's SafePOSIX construction
%has vulnerabilities.

\noindent
\textbf{Setup.}
To understand the potential security risks if Lind's SafePOSIX re-creation
has vulnerabilities, we conducted system call fuzzing with Trinity
to obtain the reachable kernel trace in Linux kernel 3.14.1.
The goal is to see how much kernel is exposed to
SafePOSIX. Since our SafePOSIX runs inside the Repy sandbox kernel,
fuzzing it suffices to determine the portion of the kernel reachable from
inside the sandbox.

\begin{table}
\centering
\scriptsize
\begin{tabular}{|l|l|l|l|l|}
  \hline
  \multirow{3}{1.5cm}{\bf Virtualization system} & \multirow{3}{0.5cm}{\bf \# of Bugs} & \multicolumn{3}{c|}{\bf Kernel trace} \\ \cline{3-5}
  & & \multirow{2}{1.5cm}{Total coverage (LOC)} & \multirow{2}{1.3cm}{In popular paths (LOC)} & \multirow{2}{1.3cm}{In risky paths (LOC)}  \\
  & & & & \\  \hline
  Lind & 1 & 70.3K & 70.3K & 0 \\
  \hline
  Repy & 1 & 74.4K & 74.4K & 0 \\
  \hline
\end{tabular}\caption{\small Reachable kernel trace analysis for Repy.}
\label{table:trace-Repy}
\end{table}

\noindent
\textbf{Results.}
The results are shown in Table \ref{table:trace-Repy}.
The trace of Repy is slightly larger (5.8\%) than that of Lind.
This larger design does not allow attackers or bugs to
access the risky paths in the OS kernel, and it leaves open only a small amount of
additional popular paths. These are added because some functions in Repy
have more capabilities for message sending and network connection than Lind's
system call interface.
For example, in Repy,
\texttt{sendmessage()} and \texttt{openconnection()}
functions could reach out to more lines of code when fuzzed. However, the kernel
trace of Repy still lies completely within the popular paths that
contain fewer kernel bugs. Thus, the Repy sandbox kernel
has only a very slim chance of triggering OS kernel bugs.

Since it is the direct point of contact with the OS kernel, in theory, the Repy
 sandbox kernel could be a weakness in the overall security coverage provided by Lind.
Nevertheless, the results above show that, even if it has a
bug or failure, the Repy kernel should not substantially increase the risk of triggering bugs.

\subsection{Practicality Evaluation}
\label{Performance-Evaluation}

The purpose of our practicality evaluation is to show that the 
``popular paths'' metric is practical in building real-world systems. Overhead is expected. 
We have not optimized our Lind prototype to try to improve performance, since that is not 
our main purpose of building the prototype. 

%\textbf{Test Purpose.}
%To measure Lind's runtime performance overhead compared to Native Linux
% when running real-world applications.
%Note that the performance of Lind was not optimized in any way before running
%these tests.

\noindent
\textbf{Setup.}
%To test the execution time overhead in Lind for running real-world applications,
We ran a few programs of different types to understand Lind's performance
impact.  All applications ran unaltered and correctly in Lind. To run the
applications, it was sufficient to just recompile the unmodified
source code using NaCl's compiler and Lind's \texttt{glibc} to call
into SafePOSIX.

To measure Lind's runtime performance overhead compared to Native Linux
 when running real-world applications,
we first compiled and ran six widely-used legacy applications:
a prime number calculator Primes 1.0,
GNU Grep 2.9, GNU Wget 1.13, GNU Coreutils 8.9,
GNU Netcat 0.7.1, and K\&R Cat.
We also ran more extensive benchmarks on two large legacy applications,
Tor 0.2.3 and Apache 2.0.64 in Lind.
%Both Tor and Apache were recompiled with NaCl's compiler and ran in Lind.
We used Tor's built-in benchmark program and Apache's benchmarking tool
\texttt{ab} to perform basic testing operations and record the execution time.

\begin{table}
\centering
\scriptsize
\begin{tabular}{|r|r|r|r|}
  \hline
  {\bf Application} & {\bf Native Code} & {\bf Lind} & {\bf Impact}  \\
  \hline
  Primes & 10000 ms & 10600 ms & 1.06x \\
  GNU Grep & 65 ms & 260 ms & 4.00x \\
  GNU Wget & 25 ms & 96 ms & 3.84x \\
  GNU Coreutils & 275 ms & 920 ms & 3.35x \\
  GNU Netcat & 780 ms & 2180 ms & 2.79x \\
  K\&R Cat & 20 ms & 125 ms & 6.25x \\
  \hline
\end{tabular}
\caption{\small Execution time performance results for six real-world applications: Native
Linux vs. Lind.}
\label{table:performance_apps}
\end{table}

\noindent
\textbf{Results.}
Table \ref{table:performance_apps} shows the runtime performance
for the six real-world applications mentioned above.
The Primes application run in Lind has a 6\% performance overhead. 
%compared to
%Native Linux. CPU bound applications, like the Primes, engender little overhead,
%because they run only in the NaCl sandbox. No system calls are
%required, and there is no need to go through the SafePOSIX interface. 
The small amount of overhead is generated by NaCl's instruction alignment at building time. 
%Another contributor to the overhead is that the instructions built by NaCl 
%have a higher rate of cache misses, which can slowdown the 
%program. 
We expect other CPU bound processes to behave similarly.

The other five applications 
%experienced slowdowns roughly ranging from 3x to 6x.
require repeated calls into SafePOSIX, and this additional SafePOSIX
computation produced the additional overhead.
%Since total execution time was limited
%to the magnitude of 10,000 ms, the
%user experience is still reasonably efficient.

\begin{table}
\centering
\scriptsize
\begin{tabular}{|r|r|r|r|}
  \hline
  {\bf Benchmark} & {\bf Native Code} & {\bf Lind} & {\bf Impact}  \\
  \hline
  Digest Tests: & & & \\
  Set & 54.80 nsec/element & 176.86 nsec/element & 3.22x \\
  Get & 42.30 nsec/element & 134.38 nsec/element & 3.17x \\
  Add & 11.69 nsec/element & 53.91 nsec/element & 4.61x \\
  IsIn & 8.24 nsec/element & 39.82 nsec/element & 4.83x \\
  \hline
  AES Tests: & & & \\
  1 Byte & 14.83 nsec/B & 36.93 nsec/B & 2.49x \\
  16 Byte & 7.45 nsec/B & 16.95 nsec/B & 2.28x \\
  1024 Byte & 6.91 nsec/B & 15.42 nsec/B & 2.23x \\
  4096 Byte & 6.96 nsec/B & 15.35 nsec/B & 2.21x \\
  8192 Byte & 6.94 nsec/B & 15.47 nsec/B & 2.23x \\
  Cell Sized & 6.81 nsec/B & 14.71 nsec/B & 2.16x \\
  \hline
  Cell Processing: & & & \\
  Inbound & 3378.18 nsec/cell & 8418.03 nsec/cell & 2.49x \\
  (per Byte) & 6.64 nsec/B & 16.54 nsec/B & - \\
  Outbound & 3384.01 nsec/cell & 8127.42 nsec/cell & 2.40x \\
  (per Byte) & 6.65 nsec/B & 15.97 nsec/B & - \\
  \hline
\end{tabular}
\caption{\small Performance results on Tor's built-in benchmark program: Native
Linux vs. Lind.}
\label{table:performance_tor}
\end{table}

A summary of the results for Tor is shown in Table \ref{table:performance_tor}. The
benchmarks focus on cryptographic operations,
which are CPU intensive, but also make system calls like \texttt{getpid}, and reads to
\texttt{/dev/urandom}.
The digest operations time the access of a map of message digests.
The AES operations time includes encryptions of several sizes and the creation of
message digests. Cell processing executes full packet encryption and decryption. In our
test, Lind slowed down these operations by 2.5x to 5x. We believe these
slowdowns are due to the increased code size produced by NaCl,
%\cappos{I'm not sure why this would be.  Does NaCl show this too?}
and the increased overhead from Lind's SafePOSIX system call interface.

Results for the Apache benchmarking tool \texttt{ab} are presented in Table \ref{table:performance_apache}.
In the set of experiments, Lind produced performance slowdowns around 2.7x.
Most of the overhead was incurred due to system call operations inside the
SafePOSIX re-creation.

Performance overhead in Lind is reasonable, considering that 
we did not specifically optimize any part of the code to improve speed. 
It should also be noted that performance slowdown is common in virtualization systems.
For example, Graphene \cite{Graphene-14} also shows an overhead ranging from
1.4x to 2x when running applications such as the Apache web server and Unixbench suite \cite{UnixBench}.
In many cases, Lind shares the same magnitude of slowdown with Graphene.
%Since an attack on the kernel can have devastating
%consequences, %at this initial stage,
%a tradeoff between security and performance could be justified.
%The fact that Lind has shown an ability to run
%\cappos{4 applications isn't many...
%Do we have Apache numbers or something else to quantify?}
%\yiwen{I am looking at more apps and libs that we can run and test in Lind.}
%legacy applications
%suggests that it is worth continuing research to optimize these systems.
Lind's ability to run a variety of programs demonstrates the practicality of our ``popular paths'' metric. 

\begin{table}
\centering
\scriptsize
\begin{tabular}{|r|r|r|r|}
  \hline
  {\bf \# of Requests} & {\bf Native Code} & {\bf Lind} & {\bf Impact}  \\
  \hline
  10 & 900 ms & 2400 ms & 2.67x \\
  20 & 1700 ms & 4700 ms & 2.76x \\
  50 & 4600 ms & 13000 ms & 2.83x\\
  100 & 10000 ms & 27000 ms & 2.70x\\
  \hline
\end{tabular}
\caption{\small Performance results on Apache benchmarking tool \texttt{ab}: Native
Linux vs. Lind.}
\label{table:performance_apache}
\end{table}
