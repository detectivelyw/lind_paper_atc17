\section{Goals and Threat Model}
\label{sec.motivation-and-background}

In this section, we define the scope of our efforts. We also briefly note why
this study does not evaluate a few existing design schemes.

\noindent
\textbf{Goals.}
Ultimately, our goal is to help designers
create systems that allow untrusted programs to
run on unpatched and vulnerable host OSes without triggering vulnerabilities that
attackers could exploit.
Developing effective defenses for the host OS kernel is essential as kernel code
can expose privileged access to attackers that could lead to a system takeover.

Our hypothesis is that OS kernel code paths that are frequently used receive
more attention and therefore are less likely to contain security
vulnerabilities. Our approach will be to test this hypothesis and explore the
feasibility of building more secure virtualization systems, such as guest OSVMs, system
call interposition modules, and library OSes, by forcing untrusted applications
to stay on popular kernel code paths.



\noindent
\textbf{Threat model.}
%We start by acknowledging that
When an attack attempt is staged
on a host OS in a virtualization system,
%Furthermore, %we anticipate that
the exploit can be done either directly or indirectly.
In a direct exploit, the attacker accesses a vulnerable portion of the host OS's kernel
using crafted attack code. In an indirect exploit,
the attacker first takes advantage of a vulnerability in the virtualization system itself
(for example, a buffer overflow vulnerability)
to escape the VM's containment. Once past the containment, the attacker can run arbitrary code
in the host OS.
The secure virtualization system design we propose
in Section~\ref{sec.design} can prevent both types of attacks effectively.

Based on the goals mentioned above, we make the following assumptions about the
potential threats our system could face:

\begin{itemize}\setlength\itemsep{0em}

\item The attacker possesses knowledge of one or more unpatched
vulnerabilities in the host OS.

\item The attacker can execute any code in the secure
virtualization system.

\item If the attack program can trigger a vulnerability in any privileged
code, whether in the host OS or the secure virtualization system, the attacker
is then considered successful in compromising the system.

\end{itemize}

%\noindent
%\textbf{Exclusion.}
%It should be noted that our study intentionally excludes
%a comparison with solutions that do not run on top of a
%full-fledged privileged OS, such as
%a bare-metal hypervisor~\cite{Xen-03, VMWare-Server} or
%hardware-based virtualization~\cite{IntelVT, keller2010nohype}.
%While our techniques can potentially apply to those
%systems, a direct comparison is not possible since they have different
%ways of accessing hardware resources, and require different measuring approaches.

%In addition, we exclude evaluation and direct comparison with full virtualization virtual machines,
%such as VirtualBox \cite{VirtualBox}, VMWare Workstation \cite{VMWare-Workstation}, and QEMU \cite{QEMU}.
%Such systems simulate hardware to allow an unmodified guest OS to run. The goal
%of our design is to substitute the large and complex TCB required for a guest OS, with a single-process
%program with a small TCB and a secure isolated environment. With different goals, our proposed design is
%a fundamentally different approach from full virtualization. As a result, direct measurement and comparison between full virtualization
%and our design is
%beyond the scope of this work.
