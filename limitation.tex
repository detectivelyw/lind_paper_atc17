\section{Limitations}
\label{sec.limitation}

One of our challenges in conducting this study was deciding where to place the
limits of its scope. To explore any one strategy
in depth, we felt it was necessary to intentionally exclude consideration of
a few other valid approaches. These choices may have placed some limitations on our results.

One limitation is that there are some types of bugs that are difficult to evaluate using our metric.
For example, bugs caused by a race condition, or that involve defects in the internal kernel data
structures, or ones that require complex triggering conditions across multiple kernel
paths, may not be immediately identified using our metric. As we continue to refine our metric,
we will also look to evolve our evaluation criteria to find and protect against more complex types of bugs.

Another limitation is that our current metric concludes that certain lines of
code in the kernel were reached or not.
Though this is an important factor in exploiting a bug, it may not be fully
sufficient for all bugs.
While a stronger conclusion about bug exploitation conditions would be ideal,
it would be hard to do so using a quantitative metric.
Instead, it would require a more complicated manual process, which was outside
the scope of this study.
