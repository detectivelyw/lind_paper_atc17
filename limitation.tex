\section{Limitation} 
\label{sec.limitation}

One of our challenges in conducting this study was deciding where to place the
limits of its scope.  To explore any one strategy
in depth, we felt it was necessary to intentionally exclude consideration of
a few other valid approaches. These choices may have placed some limitations on our results.

One such limitation stems from our chosen criteria for locating
bugs. At the beginning
of our study, we identified a set of common, but dangerous, zero-day bugs
and then we looked for them in our obtained kernel traces. By looking only
for a specific subset of bugs, we might have limited our
ability to find a broader spectrum of kernel vulnerabilities. For example, bugs
caused by a race condition, or that involve defects in the internal kernel data
structures, or ones that require complex triggering conditions across multiple kernel
paths, may not be immediately found using our metric. As we continue to refine
our metric, we will look to also evolve our evaluation
criteria to find and protect against more complex types of bugs. 
In the meantime, avoiding the triggering of this initial set of bugs
through the use of our \lip design can address the security
needs of a significant segment of users. 

Another limitation is that our current metric helps us conclude that certain lines of code in 
the kernel were reached or not, which is a necessary but may not be a sufficient condition 
to exploit a bug. While a stronger conclusion about the bug exploitation condition would be ideal, 
it would be hard to achieve by using a quantitative metric, and would require a more complicated manual process. 
