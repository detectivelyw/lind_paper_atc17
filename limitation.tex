\section{Limitation} 
\label{sec.limitation}

One of our challenges in conducting this study was deciding where to place the
limits of its scope. To explore any one strategy
in depth, we felt it was necessary to intentionally exclude consideration of
a few other valid approaches. These choices may have placed some limitations on our results.

One limitation in our evaluation of bugs is that there are some types of bugs that are hard to be evaluated by our metric. 
For example, bugs caused by a race condition, or that involve defects in the internal kernel data
structures, or ones that require complex triggering conditions across multiple kernel
paths, may not be immediately identified using our metric. As we continue to refine our metric, 
we will look to also evolve our evaluation criteria to find and protect against more complex types of bugs. 

Another limitation is that our current metric helps us conclude that certain lines of code in 
the kernel were reached or not, which is a necessary but may not be a sufficient condition 
to exploit a bug. While a stronger conclusion about the bug exploitation condition would be ideal, 
it would be hard to achieve by using a quantitative metric, and would require a more complicated manual process. 
