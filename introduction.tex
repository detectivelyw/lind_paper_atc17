\section{Introduction}
\label{sec.introduction}

The number of attacks involving the exploitation of zero-day vulnerabilities
more than doubled from 2014 to 2015 \cite{zero-day}. Skilled hackers can find a
 security flaw in a system and use it to hold the system's users hostage, e.g.,
by gaining root access and compromising the host \cite{linux-0day}. Similarly,
zero-day vulnerabilities can be exploited \cite{fbi-0day} or their presence not
be acknowledged \cite{nsa-0day} by government agencies, thus rendering millions
of devices vulnerable.

In theory, running a program in an operating-system-level virtual machine (OSVM)
like Docker \cite{Docker} or LXC \cite{LXC} should
prevent bugs in the host OS kernel from triggering.
However, the isolation provided by such systems is not the whole answer and
faces some significant drawbacks.
To be effective, OSVM's software must not contain any bugs that could allow
the program to escape the machine's containment and interact directly with the host OS.
Unfortunately, these issues are very common in OSVMs, with 14 CVE vulnerabilities
confirmed for Docker \cite{Docker-Vulnerabilities} since 2014.
The large amount of complex code needed to run such a system increases the odds
that flaws will be present, and, in turn,
that tens of millions of user machines could be at risk \cite{linux-0day}.
Furthermore, isolation will not work if a malicious program can access even a small portion of the host OS's kernel
that contains a zero-day flaw \cite{CVE-2016-5195}.
Both of these drawbacks reveal the key underlying weakness in designing OSVM systems -- a lack of information
as to which parts of the host kernel can be safely exported to user programs.

Several attempts have been made to find a reliable metric to pinpoint where bugs
 are most likely to be in kernel code.
A number of previous studies have suggested that older code may be more vulnerable
 than new code \cite{ozment2006milk}
or that certain parts (such as device drivers) of the kernel \cite{PittSFIeld}
may be more bug-prone than others.
To these hypotheses, we add a new security metric idea, called ``popular paths.''
Positing that bugs in the popular paths, associated with frequently-used programs,
are more likely to be found in software testing
because of the numerous times they are executed by diverse pieces of software,
we propose that kernel code found in these paths 
would have less chance of containing bugs than code in less-used parts of the kernel.
We perform a quantitative analysis of resilience to flaws in two versions
of the Linux kernel (version 3.13.0 and version 3.14.1),
and find that only about 3\% of the bugs are present in popular code paths,
despite these paths accounting for about one-third of the total reachable kernel code.
When we test our ``popular paths'' metric against the two aforementioned
 ``code age'' and ``device drivers'' metrics,
we find our ``popular paths'' metric is much more effective
 (Section~{\ref{Verification-of-Hypothesis}}).

This key information inspired the idea that if we could design virtual machines
 that use only ``popular kernel paths,'' a strategy we have dubbed \lip,
it would greatly increase resilience to zero-day bugs in the host OS kernel.
Yet using such a design scheme creates a few challenges that would need to be
 overcome. These include:

\begin{itemize}
\item It might not be possible in real-life codebases to completely avoid ``unpopular paths.''
If other applications, or future versions of the applications we tested, frequently require the use of ``unpopular paths,'' would this make our metric untenable?
(Section~{\ref{lock-in-pop}})
\item The exploits that adversaries use change over time. Could our observation that ``popular paths'' are safer be only an artifact of when we did our measurements,
and not be predictive of future exploits? (Section~{\ref{Verification-of-Hypothesis}})
\item Lastly, can developers make use of this observation in a practical setting? That is, is it feasible for developers to actively try to avoid unpopular code paths?
(Section~{\ref{implementation}})
\end{itemize}

While we address some of these challenges in developing the \lip design,
we want to test how well a system could function if it forced applications to
use only popular kernel paths.
To conduct these tests, we built a prototype system, called Lind.
For Lind, we pick two key components -- Google's Native Client
(NaCl) \cite{NaCl-09} and Seattle's Repy \cite{Repy-10}.
NaCl serves as a computational module that isolates
binaries, providing memory safety for legacy programs running in our OSVM.
It also passes system calls invoked by the program to the operating system interface, called SafePOSIX.
SafePOSIX re-creates the broader POSIX functionalities needed by applications, while being contained within the Repy sandbox.
An API in the sandbox only allows access to popular kernel paths, while
the small (8K LOC) sandbox kernel of Repy isolates flaws in SafePOSIX
to prevent them from allowing direct access to the host OS kernel.

To test the effectiveness of Lind and our ``popular paths'' metric,
we replicated 35 kernel bugs discovered in Linux kernel version 3.14.1. We attempted
to trigger those bugs in Lind and three other virtualized environments,
including Docker~\cite{Docker}, LXC~\cite{LXC}, and Graphene~\cite{Graphene-14}.
In this study, our evaluation was focused on comparing operating-system-level virtualization containers, such as Docker and LXC,
and library OSes, such as Graphene.
We excluded bare-metal hypervisors~\cite{Xen-03, VMWare-Server},
hardware-based virtualization~\cite{IntelVT, keller2010nohype} and full virtualization
virtual machines, such as VirtualBox \cite{VirtualBox}, VMWare Workstation \cite{VMWare-Workstation}, and QEMU \cite{QEMU}.
While our ``popular paths'' metric may potentially apply to those
systems, a direct comparison is not possible since they have different
ways of accessing hardware resources, and would require different measurement approaches.

Our results show that applications in Lind are substantially less likely to trigger kernel bugs.
By doing so, we demonstrate that forcing an application to use only popular
OS paths can be an effective and practical method to improve
system security. Armed with this knowledge, the \lip principle can be adapted to
incorporate other OSVM design configurations.

In summary, the main contributions of this paper are as follows:

\begin{itemize}\setlength\itemsep{0em}
\item
We propose a quantitative metric that evaluates security at the line-of-code level,
and verify our hypothesis that ``popular paths'' have significantly fewer security bugs than other paths.
\item
Based on the ``popular paths'' metric, we develop a new design scheme called \lip that accesses only popular code paths through a very small trusted computing base.
The need for complex functionality is addressed by re-creating riskier system calls in a memory-safe programming language within a secure sandbox.
\item
To demonstrate the practicality of the ``popular paths'' metric, we build a prototype virtual machine, Lind, using the \lip design,
and test its effectiveness against three other virtual machines. We find that Lind exposes 8-12x fewer zero-day kernel bugs.
\end{itemize}
