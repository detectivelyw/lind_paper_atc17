\section*{Abstract}

%Virtual machines (VMs) are widely used in practice, in part for their ability to
%isolate potentially untrusted code from the rest of a system.
%Recently, library OSes and containers have also presented promising security options.
%
%However, it is often possible to trigger zero-day flaws
%in the host Operating System (OS) from inside of such virtualized systems.
%

%In this paper, we offer a new insight about where security bugs lie. By observing that the OS kernel paths accessed
%by popular applications in everyday use contain significantly fewer security bugs than less-used paths,
%we devise a design that allows applications to run more securely in VMs on top of a vulnerable host OS.
%Furthermore, We
%leverage this observation to devise the \lip design, which
%\textbf{\textit{locks}} an application, and the POSIX implementation that services it, into
%accessing only the well-used \textbf{\textit{popular}} portion of the kernel.  Using the \lip model, we
%implement a prototype virtual machine called Lind.
%
%We compare Lind and three other virtualized systems that were
%available at the release of Linux kernel version 3.14.1, and evaluate
%their effectiveness in containing the zero-day kernel bugs that have been discovered
%since then.
%
%Our results show that Lind can prevent the triggering of zero-day kernel bugs significantly better
%than an existing library OS (Graphene) and containers such as Docker and LXC.

Virtual machines (VMs) that try to isolate untrusted code are widely used in practice.
However, it is often possible to trigger zero-day flaws
in the host Operating System (OS) from inside of such virtualized systems.
%
In this paper, we propose a new security metric showing strong correlation
between ``popular paths''
and kernel vulnerabilities. We verify that the OS kernel paths accessed
by popular applications in everyday use contain significantly fewer security
bugs than less-used paths. We then demonstrate that this observation is
useful in practice by building a prototype system which \textbf{\textit{locks}}
an application into using only \textbf{\textit{popular}} OS kernel paths.
By doing so, we demonstrate that we can prevent the triggering of zero-day
kernel bugs significantly better than three other competing approaches, and
argue that this is a practical approach to  secure system design.
